\begin{abstract}
    Learning to recontruct 2.5D geometry from video in an unsupervised manner with deep convolutional network (DCN) \cite{,} is acctracting significant attention in recent years, 
    due to that it can be widely applied to unlabeled videos online for applications such as augment relality etc.
    In this paper, we propose to recover depth by introducing an internal normal representation in an unified unsupervised pipeline, where the estimation of depth is strongly regularized by the predicted normals, yielding more robust and consistent estimation for both depth and normals. 
    Specifically, we introduce two layers, i.e. a depth to normal layer and a normal to depth layer.
    The depth to normal layer takes estimated depth for each pixel as input, and compute normal directions with inner product. Then given the normal and depth, the normal to depth layer outputs a regularized depth through local planar smoothness.
    Finally, to train the network we apply the photometric loss as proposed in~\cite{}, and further require gradient smoothness for both depth and normals predictions. 
    We conducted experiments on both outdoor (KITTI) and indoor (NYUv2) videos, and show that our algorithm vastly outperform the state-of-arts, reducing both depth and normal error over 20$\%$ relatively, which demonstrates the benefits from the normal representation.
\end{abstract}
